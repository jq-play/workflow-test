\section{Trajectory}
\label{sec:Trajectory}
%%%%%%%%%%%%%%%%%%%%%%%%%%%%%%%%%%%%%%%%%%%%%%%%%%%%%%%%
\begin{figure}[h!]
\begin{center}
\includegraphics[width=.8\textwidth]{figures/trajectory}
\end{center}
\caption{Schema Diagram for the Trajectory Pattern. The visual notation is explained in Chapter \ref{chap:prelims}.}
\label{fig:Spatiotemporal}
\label{fig:Trajectory}
\end{figure}
\subsection{Summary}
\label{sum:Trajectory}
%%%%%%%%%%%%%%%%%%%%%%%%%%%%
The \textsf{Trajectory} Pattern allows a developer to track something moving through some space. This is, of course, very abstract and is intended to be a starting point for capturing any movement that occurs at discrete points in a space. Intuitively, there is the notion of moving through time and space and those captured discrete points in space may be GPS position recordings. This sort of data may be best captured with the \textsf{SpatiotemporalExtent} Pattern (Section \ref{sec:Spatiotemporal}), which extends the \textsf{Trajectory} Pattern. This pattern may be also used as a starting point for modelling procedures (i.e. steps are discrete points in procedure space) or chemical reactions (we can really only be sure of what our sensors tell us, and they only tell us things at their polling rates). This pattern is an abstraction of the Semantic Trajectory pattern found in \cite{traj}.

%%%%%%%%%%%%%%%%%%%%%%%%%%%%%%%%%%%%%%%%%%%%%%%%%%%%%%%%
\subsection{Axiomatization}
\label{axs:Trajectory}
%%%%%%%%%%%%%%%%%%%%%%%%%%%%
\begin{align}
\textsf{Segment} &\sqsubseteq \mathord{=1}\textsf{startsFrom.Fix}\\
\textsf{Segment} &\sqsubseteq \mathord{=1}\textsf{endsAt.Fix} \\
\textsf{Segment} &\sqsubseteq \exists \textsf{hasSegment}^-\textsf{.Trajectory} \\
\textsf{startsFrom}^- \circ \textsf{endsAt} &\sqsubseteq \textsf{hasNext} \\
\textsf{hasNext} &\sqsubseteq \textsf{hasSuccessor} \\
\textsf{hasSuccessor} \circ \textsf{hasSucessor} &\sqsubseteq \textsf{hasSucessor} \\
\textsf{hasNext}^- &\equiv \textsf{hasPrevious} \\
\textsf{hasSuccessor}^- &\equiv \textsf{hasPredecessor} \\
\textsf{Fix} \sqcap \lnot\exists\textsf{endsAt}^-\textsf{.Segment} &\sqsubseteq \textsf{StartingFix} \\
\textsf{Fix} \sqcap \lnot\exists\textsf{startsFrom}^-\textsf{.Segment} &\sqsubseteq \textsf{EndingFix} \\
\textsf{Trajectory} &\sqsubseteq \exists\textsf{hasSegment.Segment} \\
\textsf{hasSegment} \circ \textsf{startsFrom} &\sqsubseteq \textsf{hasFix} \\
\textsf{hasSegment} \circ \textsf{endsAt} &\sqsubseteq \textsf{hasFix} \\
\exists \textsf{hasSegment.Segment} &\sqsubseteq \textsf{Trajectory} \\
\exists \textsf{hasSegment}^-\textsf{.Trajectory} &\sqsubseteq \textsf{Segment} \\
\exists \textsf{hasFix.Segment} &\sqsubseteq \textsf{Trajectory} \\
\exists \textsf{hasFix}^-\textsf{.Trajectory} &\sqsubseteq \textsf{Fix}
\end{align}

%%%%%%%%%%%%%%%%%%%%%%%%%%%%%%%%%%%%%%%%%%%%%%%%%%%%%%%%
\subsection{Explanations}
\label{exp:Trajectory}
%%%%%%%%%%%%%%%%%%%%%%%%%%%%
\begin{enumerate}
\item \textsf{Segment} \textsf{startFrom} exactly one \textsf{Fix}.
\item \textsf{Segment} \textsf{endsAt} exactly one \textsf{Fix}.
\item Existential: A \textsf{Segment} belongs to at least one \textsf{Trajectory}.
\item Role Chain: the concatenation of \textsf{startsFrom}$^-$ nad \textsf{endsAt} is \textsf{hasNext}.
\item Subproperty: \textsf{hasNext} is a subproperty to \textsf{hasSuccessor}.
\item Role Chain: \textsf{hasSuccessor} is transitive.
\item Inverse Alias.
\item Inverse Alias.
\item A \textsf{Fix} that is not where a segment ends is a \textsf{StartingFix}.
\item A \textsf{Fix} that is not where a segment starts is a \textsf{EndingFix}.
\item Existential: a \textsf{Trajectory} has at least one \textsf{Segment}.
\item Role Chain: the concatenation of \textsf{hasSegment} and \textsf{startsFrom} is \textsf{hasFix}.
\item Role Chain: the concatenation of \textsf{hasSegment} and \textsf{endsAt} is \textsf{hasFix}.
\item Scoped Domain: the domain of \textsf{hasSegment}, scoped by \textsf{Segment}, is \textsf{Trajectory}.
\item Scoped Domain: the domain of \textsf{hasSegment}$^-$, scoped by \textsf{Trajectory}, is \textsf{Segment}.
\item Scoped Domain: the domain of \textsf{hasFix}, scoped by \textsf{Segment}, is \textsf{Trajectory}.
\item Scoped Domain: the domain of \textsf{hasFix}$^-$, scoped by \textsf{Trajectory}, is \textsf{Fix}.
\end{enumerate}

%%%%%%%%%%%%%%%%%%%%%%%%%%%%%%%%%%%%%%%%%%%%%%%%%%%%%%%%
\subsection{Competency Questions}
\label{cqs:Trajectory}
%%%%%%%%%%%%%%%%%%%%%%%%%%%%
\begin{enumerate}[CQ1.]
\item What is the first step of the procedure?
\item What was the cruise's final stop?
\end{enumerate}

\newpage
%%%%%%%%%%%%%%%%%%%%%%%%%%%%%%%%%%%%%%%%%%%%%%%%%%%%%%%%
% End Section
%%%%%%%%%%%%%%%%%%%%%%%%%%%%%%%%%%%%%%%%%%%%%%%%%%%%%%%%
%%%%%%%%%%%%%%%%%%%%%%%%%%%%%%%%%%%%%%%%%%%%%%%%%%%%%%%%